\section{DDOS protection Services Provider (DPS)}\label{sec:DPS}
DDoS Protection Service (DPS) providers mitigating DDoS attacks through scrubbing centres, DPS can be categorized into BGP-based and DNS-based services BGP-based DPS \cite{tung2018understanding}. is favoured for large-scale attacks, leveraging the Border Gateway Protocol to reroute overwhelming malicious traffic from the victim's network to more resilient networks, significantly reducing the attack's impact.

Conversely, DNS-based DPS utilizes DNS redirection, altering the DNS records of the target domain to reroute traffic to the DPS provider’s servers. This method filters traffic, removing malicious activities and ensuring only legitimate traffic reaches the target. It hinges on changing DNS resolution, directing domain requests to DPS servers instead of the target's original servers.
new-branch-name
\section{BGP-Based DPS: Benefits and Limitations}\label{sec:BDPS_benefits}
BGP-based DPS excels in mitigating large-scale volumetric attacks. Utilizing the Border Gateway Protocol, it diverts malicious traffic from the victim's network to robust networks equipped for traffic scrubbing. This strategy effectively diminishes the attack's impact and upholds the integrity of the target network, making it a preferred solution for defending against significant DDoS threats. However, implementing BGP-based DPS requires a substantial network infrastructure, typically involving extensive connectivity through multiple transits, Private Network Interconnects (PNI) \cite{chatzis2013there}., and Network Access Points (NAPs), leading to significant investment requirements. Moreover, there is a minimum network size constraint: the network must possess at least a Class C /24 prefix for public internet propagation, as this is the smallest network prefix that can be propagated on the public internet via BGP routing \cite{caesar2005bgp}. Smaller networks, which do not have a /24 network or larger, might find this requirement challenging, limiting the feasibility of BGP-based DPS for such organizations.

\section{DNS-based DPS: Benefits and Limitations}\label{sec:DDPS_benefits}
DNS-based DDoS Protection Services (DPS) offer effective mitigation strategies by modifying the address record (A-record) in the target's DNS server \cite{schomp2020akamai}. The strategic adjustment of the Time-To-Live (TTL) values facilitates rapid updates and propagation of DNS records, which is essential in rerouting traffic efficiently during an attack. This method proves especially beneficial for web portals or applications dependent on DNS for reachability, offering a viable solution for small and medium-sized enterprises (SMEs) or organizations with smaller network infrastructures, as it does not necessitate a /24 network prefix. However, the effectiveness of DNS-based DPS is constrained when facing direct-to-IP attacks. Such attacks include services like FTP (File Transfer Protocol), SSH, SMTP (Simple Mail Transfer Protocol) for email, or other proprietary protocols often used in enterprise applications. Given that these services typically operate using fixed IP addresses rather than DNS lookups, DNS-based DPS may struggle to adequately mitigate attacks targeting them.


\section{Operational Strategies: Detection and Mitigation}\label{sec:operations}
BGP-Based DPS operational strategies are designed to address high-volume attacks targeting network bandwidth. These services utilize a comprehensive detection approach that encompasses Deep Packet Inspection (DPI), netflow analysis, and traffic graph examination \cite{wang2018delving} \cite{saied2016detection}. DPI, including critical payload inspection, is essential in accurately identifying Layer 3 and 4 attacks and is central to attack traffic fingerprinting, providing in-depth information crucial for the mitigation process. DPI's ability to delve into the specifics of packet content makes it an invaluable tool in distinguishing between attack and legitimate traffic. Complementing DPI, netflow and traffic graphs offer insights into traffic flow characteristics and aid in profiling normal legitimate user traffic. This helps in detecting anomalous patterns indicative of DDoS activities. Nonetheless, DPI strategies come with limitations, including their complexity, computational demands, and challenges in analyzing encrypted traffic. In contrast, DNS-Based DPS targets application-layer attacks and emphasizes analyzing content requests \cite{xie2008monitoring}. Unlike Layer 3 or 4 attacks where DPI provides significant benefits, application-layer attack detection in DNS-based DPS can rely on sufficient telemetry from traffic graphs and netflow analysis without necessitating DPI. This is particularly relevant when the network traffic is unrelated to the application layer, such as UDP packet floods or TCP packets on unrelated ports, which are not pertinent for a targeted web server's DPI analysis. DNS-based DPS instead places a greater emphasis on application logs from targeted systems. These logs are pivotal in identifying activities that exploit application vulnerabilities or functionalities, offering direct insights into how the application is being manipulated or overwhelmed by attack traffic. While DPI offers thorough traffic analysis, its effectiveness for DNS-based DPS is limited due to the complexities and demands of DPI, and its limited utility in encrypted traffic analysis.

\section{The telemetry discrepancy among DPS in the industry}\label{sec:telemetry}
The telemetry discrepancy between BGP-Based and DNS-Based DDoS Protection Services (DPS) is primarily attributed to the difference in detection and mitigation strategies across various OSI layers and the distinct nature of their customer bases. 
\paragraph{Scope and Target Audience Differences} BGP-Based DPS, focusing on network prefixes, caters predominantly to large corporations and enterprises. Within a single network prefix, BGP-Based DPS addresses a multitude of entities spanning different network, transport, and application protocols. Conversely, DNS-Based DPS typically provides services per application, often associated with a single IP address, as outlined in the operational strategies section. 

\paragraph{Balancing Volumetric Protection and Application Layer Control} Additionally, customers utilizing BGP-Based DPS, such as financial institutions, government bodies, and banks, may seek volumetric protection while retaining control over their application layer. For instance, banking customers might be reluctant to share SSL certificates for their portals with the DPS, preferring to maintain exclusive control over the decryption of application content and requests.
\section{Industry Trends: Offering Hybrid DPS Solutions}\label{sec:solutions}
Some DPS providers in the industry are now offering both BGP-based and DNS-based services to provide a comprehensive DDoS mitigation portfolio. This hybrid approach allows for a more versatile defense strategy, catering to a broader range of attack vectors and offering tailored solutions based on the specific needs and infrastructure of the client.