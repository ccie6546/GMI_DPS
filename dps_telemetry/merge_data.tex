\section{Mapping victims with Telescopes and Honeypot}\label{sec:mapping_victim}
This analysis establish a relationship between our recorded DDOS event data and the Telescopes/Honeypots datasets. It specifically focuses on aligning full-day dates, spanning from midnight to the end of the day, with our defined intervals of DDOS events, as marked by startTime and endTime. 
In Telescopes' operational design, attacks are detected by analyzing backscattered traffic. This traffic typically results from attack traffic that spoofs its source address to resemble that of the Telescopes' address blocks. Based on this detection method, we can anticipate two primary scenarios:
Due to our mitigation mechanisms, there is only a slim chance that the date recorded by Telescopes will coincide with our event dataset. If Telescopes record a date that precedes the startTime and endTime of a DDOS event, it might be indicative of an attack being detected early by the Telescopes. Following this early detection, mitigation measures might be activated within the DDOS event window to address the attack. On the other hand, if a date recorded by Telescopes falls after the DDOS event timeframe, it could imply a re-emergence of the attack, occurring after the mitigation measures detailed in the DDOS event data.
The Honeypots dataset, however, provides a contrasting perspective. Unlike Telescopes, Honeypots are not just detection tools; they actively participate in attack mechanisms. As such, their data might align with DDOS events by coincidence. The occurrence of a Honeypot record during a DDOS event does not inherently suggest a direct link with the event's mitigation processes, as Honeypots operate independently of these countermeasures

\section{Date Alignment}\label{sec:date_events_alignment}
The data from the telescopes, which aligns with our DDOS events, is vividly represented by the spikes in True counts shown in the graph. 
Customer\_ip\_matched in this context identifies IP addresses of our customers that are detected and logged within the telescopes/honeypot datasets. The designation `customer,' however, is more specifically applied to those customer IPs that not only appear in the telescopes/honeypot datasets but also coincide with the broader timeframes delineated by the \texttt{startTime} and \texttt{endTime} of our DDOS events. Crucially, this correlation is not based on exact hour, minute, and second details, since the telescopes/honeypot data does not include these precise time elements. Rather, matching is determined by the day, with the assumption that each recorded date encompasses the full 24-hour span from one midnight to the next.
